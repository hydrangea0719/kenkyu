% kenkyukai_template.tex を編集

%\documentstyle[epsf,twocolumn]{jarticle}       %LaTeX2.09仕様
\documentclass[twocolumn]{jarticle}
%%%%%%%%%%%%%%%%%%%%%%%%%%%%%%%%%%%%%%%%%%%%%%%%%%%%%%%%%%%%%%
%%
%%  基本 バージョン
%%
%%%%%%%%%%%%%%%%%%%%%%%%%%%%%%%%%%%%%%%%%%%%%%%%%%%%%%%%%%%%%%%%
\setlength{\topmargin}{-45pt}
%\setlength{\oddsidemargin}{0cm}
\setlength{\oddsidemargin}{-7.5mm}
%\setlength{\evensidemargin}{0cm}
\setlength{\textheight}{24.1cm}
%setlength{\textheight}{25cm}
\setlength{\textwidth}{17.4cm}
%\setlength{\textwidth}{172mm}
\setlength{\columnsep}{11mm}


%【節がかわるごとに(1.1)(1.2) …(2.1)(2.2)と数式番号をつけるとき】
%\makeatletter
%\renewcommand{\theequation}{%
%\thesection.\arabic{equation}} %\@addtoreset{equation}{section}
%\makeatother

%\renewcommand{\arraystretch}{0.95} 行間の設定

%%%%%%%%%%%%%%%%%%%%%%%%%%%%%%%%%%%%%%%%%%%%%%%%%%%%%%%%
\usepackage{graphicx}   %pLaTeX2e仕様(要\documentstyle ->\documentclass)
%%%%%%%%%%%%%%%%%%%%%%%%%%%%%%%%%%%%%%%%%%%%%%%%%%%%%%%%

\begin{document}

\twocolumn[
\noindent

\hspace{1em}

情報工学実験I\hspace{-.1em}I
\hfill
\ \ B3 多田 瑞葵


\vspace{2mm}

\hrule

\begin{center}
{\Large \bf BERT を用いた深層言語処理における品詞推定の調査}
\end{center}


\hrule
\vspace{3mm}
]

\section{はじめに}
近年,人工知能や深層学習による創作に関する研究が盛んになっている.自然言語処理の分野においても小説の生成を目指した文や文章の生成などが研究されているが,これには未だ様々な課題がある.その一つに,生成される文の文法情報が正しく得られるかどうかという課題がある.\par
本実験では,最近注目を集めている深層言語モデル BERT を用いて,文の品詞の情報がどの程度獲得できているかを実験をもとに調査した.


\section{要素技術}
% 要素技術と従来手法をあわせて要素技術

  \subsection{BERT}
BERT (Bidirectional Encoder Representations from Transformers) \cite{DBLP} は,2018 年 10 月に Google の Jacob Devlin らが発表した,Transformer による双方向のエンコーダーを用いた言語モデルである.従来の言語モデルは特定の自然言語処理タスクにのみ対応しており,タスクに応じてモデルの修正が必要であった.しかし BERT はファインチューニングをすることで,モデルの修正をせずに様々なタスクに応用でき,汎化性に優れている.また当時 11 個のタスクに対して state-of-the-art を達成しており精度が高い.\par
BERT は事前学習モデルであり,入力文の一部の単語を "[MASK]" 記号に置き換えてその元単語を予測するタスクと,2 つの入力文に対して文の連続性を識別するタスクによって学習する.本実験では,東北大学から公開されている,日本語 Wikipedia の文章を用いて事前学習されたモデル\footnote{https://www.nlp.ecei.tohoku.ac.jp/news-release/3284/} を使用した.


  % BERT は一部の単語を [MASK] に置き換えることで,そこの単語を予測してくれます.そもそも BERT は二種類の事前学習をしていて,単語の予測と文の関係性の予測の二つを勉強していて,それによってとても賢いです.今回はその単語の予測の部分(BERT で事前学習している部分)を使うのですが,事前学習とタスクが同じなので,ファインチューニングが不要ですか...?(何もわかっていない顔)


% 脚注の入れ方
%http://www.latex-cmd.com/struct/footnote.html

  \subsection{MeCab}
  MeCab\footnote{https://taku910.github.io/mecab/} は,オープンソース形態素解析エンジンであり,京都大学情報学研究科-日本電信電話株式会社コミュニケーション科学基礎研究所共同研究ユニットプロジェクトを通して開発された.
  \par
  日本語は,文中で単語が区切りなく並べられているため,BERT を使う前に文を単語に区切る必要がある.書くことがない!!!!!!!!!!
  
  東北大学から公開されている BERT の日本語モデルは,形態素解析器に MeCab を用いて事前学習している.そのため本実験でも形態素解析に MeCab を用いた.
  % (HPに書いてあったのをそのまま書いただけで,しかもこの〜〜プロジェクトのリンクが切れてるのでまたちゃんと調べておく)

  % また,形態素解析器はたくさんあるのだけれど,本実験で使用する BERT は MeCab を用いたトークナイズ(=形態素解析?)によって訓練されているため,同様に MeCab を使用する.
  \par
  うーーーんなんかいい感じに書く.!!!!!!



\section{使用データセット}
本実験では,叙述的な文を対象に品詞を推定するため,毎日新聞データセット\footnote{https://www.nichigai.co.jp/sales/mainichi/mainichi-data.html}の新聞記事を用いた.2008 年の 1 面に掲載された記事の中から文として成立しているものを抜き出し,それぞれの文章に対して "□","×" などの記号,"<>" などの間に書かれた注釈等を取り除いた. 
 



%\section{実験手法}
% 従来手法があるなら提案手法とする

\section{数値実験}
本実験では,文中の単語を "[MASK]" 記号に置き換え,元単語が各品詞(名詞,動詞,形容詞)であるかどうかを正しく出力できるかを調査した.\par
前処理として,データセットの各文に対して分かち書きをし,単語に分割した.文中の単語の中から 1 つをランダムに選択して "[MASK]" 記号に置き換えた.置き換えた箇所の元単語が名詞である文にラベル 1 を,そうでない文にラベル 0 を付与した.動詞,形容詞についてもそれぞれ同様に処理した.次に,データの偏りによる精度の差を避けるため,ラベル 0 とラベル 1 のデータ数を同じ 4500 個に揃えた.\par
各文を BERT に入力して,得られた各単語の分散表現を多層パーセプトロン(MLP)に入力して分類した.表 \ref{tb:1} に本実験で用いた MLP のパラメータを示す.

\begin{table}[htb]
	\caption{実験で用いた MLP のパラメータ}
	\begin{tabular}{| c |  c |} \hline
  パラメータ & 値  \\ \hline \hline
  入力層の次元数	& 分散表現の次元数 \\ %\cline{2-2}
  隠れ層の次元数	& 768 \\ 
  出力層の次元数	& 2 \\ 
  バッチサイズ		& 2 \\
  損失関数		& cross entropy \\
  最適化関数		& Adam \\
  学習率			& $5\times10^{-5}$ \\ \hline
	\end{tabular}
	\label{tb:1}
\end{table}





また,データ数が少ないため,訓練データに対して 5 分割交差検証による平均値を確認して評価した.

% \subsection{実験}




\subsection{実験結果}
表 \ref{tb:2} にそれぞれのデータセットに対して 5 分割交差検証をした際の実験精度を示す.


\begin{table}[htb]
\caption{実験精度}
  \begin{tabular}{| c |  c | c |} \hline
    品詞 & Accuracy & 平均値(標準誤差) \\ \hline \hline
    		& 0.9017 &  \\ %\cline{2-2}
    		& 0.9000 &  \\ 
 名詞  	& 0.8900 & 0.8942 \\ 
          & 0.8917 & (0.0056) \\ 
       	& 0.8878 &  \\ \hline
%
              & 0.9539 &  \\ %\cline{2-2}  
       	& 0.9600 &  \\ 
    動詞 	& 0.9533 & 0.9564 \\  
          & 0.9522 & (0.0042) \\  
       	& 0.9628 &  \\ \hline
%
              & a &  \\   
       	& a &  \\  
    形容詞 	& a &  \\  
          & a &  \\ 
       	& a &  \\ \hline
%
    ベースライン & 0.5000 & 0.5000 \\  \hline
  \end{tabular}
  \label{tb:2}
\end{table}



 \subsection{考察}
 実験の結果から,名詞,動詞,形容詞のいずれにおいても,高い精度で品詞の推定ができた.

\section{まとめと今後の課題}
本実験では,文中の単語がある品詞かそうでないかの 2 クラスに分類し,その分類精度を確認した.結果,名詞,動詞,形容詞のいずれにおいても,高い精度で品詞を推定できることが分かった.
\par
そのほかになんか書くことがあれば書く!!!!!!
\par
今後の課題として,

% 参考文献出力
\bibliography{index}
\bibliographystyle{unsrt}


\end{document}
