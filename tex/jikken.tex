% kenkyukai_template.tex を編集

%\documentstyle[epsf,twocolumn]{jarticle}       %LaTeX2.09仕様
\documentclass[twocolumn]{jarticle}
%%%%%%%%%%%%%%%%%%%%%%%%%%%%%%%%%%%%%%%%%%%%%%%%%%%%%%%%%%%%%%
%%
%%  基本 バージョン
%%
%%%%%%%%%%%%%%%%%%%%%%%%%%%%%%%%%%%%%%%%%%%%%%%%%%%%%%%%%%%%%%%%
\setlength{\topmargin}{-45pt}
%\setlength{\oddsidemargin}{0cm}
\setlength{\oddsidemargin}{-7.5mm}
%\setlength{\evensidemargin}{0cm}
\setlength{\textheight}{24.1cm}
%setlength{\textheight}{25cm}
\setlength{\textwidth}{17.4cm}
%\setlength{\textwidth}{172mm}
\setlength{\columnsep}{11mm}


%【節がかわるごとに(1.1)(1.2) …(2.1)(2.2)と数式番号をつけるとき】
%\makeatletter
%\renewcommand{\theequation}{%
%\thesection.\arabic{equation}} %\@addtoreset{equation}{section}
%\makeatother

%\renewcommand{\arraystretch}{0.95} 行間の設定

%%%%%%%%%%%%%%%%%%%%%%%%%%%%%%%%%%%%%%%%%%%%%%%%%%%%%%%%
\usepackage{graphicx}   %pLaTeX2e仕様(要\documentstyle ->\documentclass)
%%%%%%%%%%%%%%%%%%%%%%%%%%%%%%%%%%%%%%%%%%%%%%%%%%%%%%%%

\begin{document}

\twocolumn[
\noindent

\hspace{1em}

情報工学実験I\hspace{-.1em}I
\hfill
\ \ B3 多田 瑞葵


\vspace{2mm}

\hrule

\begin{center}
{\Large \bf タイトル}
\end{center}


\hrule
\vspace{3mm}
]

\section{はじめに}
文章の生成をしたいのでとりあえず二値分類から始める.
BERT が賢すぎて嬉しい.

\section{要素技術}
% 要素技術と従来手法をあわせて要素技術
  \subsection{BERT}
  BERT\cite{DBLP} は Transformer による双方向のエンコーダーである.2008 年 10 月に Google の Jacob Devlin らの論文で発表された自然言語処理モデルである.
  \par
  使うのはたぶん東北大 BERT だと思います.[MASK] をかけることで,そこの単語を予測してくれます.そもそも BERT は二種類の事前学習をしていて,単語の予測と文の関係性の予測の二つを勉強していて,それによってとても賢いです.今回はその単語の予測の部分を使うのですが,事前学習とタスクが同じなので,ファインチューニングが不要ですか...?(何もわかっていない顔)
  \subsection{MeCab}
  分かち書きをしてくれました.ありがとう.形態素解析で助詞とか名詞とかを抽出して [MASK] をかけるのに使います.

\section{実験手法}
% 従来手法があるなら提案手法とする
今回は,助詞に [MASK] をかけて,そこに入る単語が主格かどうかを判定してもらいます.他には,特定の品詞に [MASK] をかけて,それがその品詞かどうかを判定できたら嬉しいです.

\section{数値実験}
  \subsection{実験1}
  パラメータがからころちん

  \subsection{実験結果}
  実験結果が出ればここに書けますが,でなければ書けません.

  \subsection{考察}
  うーん,ばなな!

\section{まとめと今後の課題}
これからも頑張りたいです,まる.


% 参考文献出力
\bibliography{index}
\bibliographystyle{unsrt}


\end{document}
